\documentclass[11pt,a4paper,bookmarksopen=true]{article}

\usepackage{bookmark}
\usepackage[margin=1in]{geometry}
\usepackage{longtable}
\usepackage[bitheight=6ex]{bytefield}
\usepackage{color}
\usepackage{hyperref}
\usepackage{tikz}
\usetikzlibrary{arrows,shadows} % for pgf-umlsd
\usepackage[underline=true,rounded corners=false]{pgf-umlsd}
\usepackage[grumpy]{gitinfo}
\usepackage{fancyhdr}

\pagestyle{fancy}
\fancyhf{}
\renewcommand{\headrulewidth}{0pt}
\fancyfoot[L]{
  Revision\gitVtags:
  \href{https://github.com/JanAhrens/threema-protocol-analysis/commit/\gitHash}{\gitAbbrevHash}
  (\gitAuthorDate)
}
\fancyfoot[R]{\thepage}
\setlength{\parindent}{1em}

\title{Threema protocol analysis}

% poor man's spam protection
\newcommand{\atgmail}{@gmail.com}
\author{
  Jan Ahrens \\
  \\
  \begin{tabular}{r l}
    E-mail  & \texttt{jan.ahrens\atgmail} \\
    PGP key & \texttt{\href{http://pool.sks-keyservers.net/pks/lookup?op=get\&search=0xB911E6A22B4F3B5F}{3762 1152 E099 AB27 04E8}} \\
            & \texttt{\href{http://pool.sks-keyservers.net/pks/lookup?op=get\&search=0xB911E6A22B4F3B5F}{3FD1 B911 E6A2 2B4F 3B5F}}
  \end{tabular}}
\date{\gitAuthorDate}

\hypersetup{
  pdftitle={Threema protocol analysis},
  pdfauthor={Jan Ahrens},
  pdfsubject={},
  pdfkeywords={}}

% Set up hyperlink colors
\definecolor{darkred}{rgb}{0.5,0,0}
\definecolor{darkgreen}{rgb}{0,0.3,0}
\definecolor{darkblue}{rgb}{0,0,0.5}
\definecolor{darkbrown}{rgb}{0.28,0.07,0.07}
\hypersetup{
  colorlinks=true,
  citecolor=darkblue,
  urlcolor=darkgreen,
  linkcolor=darkred,
  menucolor=darkbrown}

\begin{document}

\maketitle
\vspace*{10em}

\begin{abstract}

\href{https://threema.ch/en/}{Threema} is a commercial mobile
messaging application developed by the Swiss-based company
\href{http://www.kaspersystems.ch/}{Kasper Systems GmbH}.  Its
popularity is based on the promise that each message is transferred
using end-to-end encryption. While claiming to use the open-source
cryptographic library
\href{https://threema.ch/en/faq.html#why\_secure}{NaCl}, the details
of the used protocol are closed and can not be independently reviewed and
verified. The \href{https://threema.ch/validation/}{Validation
  Logging} functionality provided by Threema does not prove that
logged messages are exchanged with the server.

In this paper I will describe the Threema protocol, with the
intention of enabling an independent review. This paper does not judge
whether there are weaknesses in the protocol or the application
itself.

The results are based on the implementation used by the Threema
Android application version 1.3.  Neither its keys
nor servers are included in this paper. They are sold by Kasper
System GmbH in form of the Threema binary.
\end{abstract}

\newpage

\tableofcontents

\section{Introduction}
Threema is a commercial mobile messaging application for Android and
iOS. Its features include chatting with contacts, group conversations and
sharing images. All communication is said to be
end-to-end encrypted using a keypair generated by the application on
its first launch.

Threema is sold by the Swiss-based company Kasper Systems GmbH
on \href{https://threema.ch/en}{its website}, through the Google
Play Store and the Apple App Store. Threema's source code is closed
and can not be reviewed without the agreement of its authors.
The application includes a
\href{https://threema.ch/validation/}{Validation Logging} feature that
can be used to log the in- and outgoing messages.  This feature
logs the message bodies, but not all the data that is exchanged with
the server.  There is no way to prove that the message that is being
logged correlates to data that is being sent by the application.
To enable an independent review I will describe the details of the
custom protocol, that Threema is using,
in this paper. The results are based on an analysis of the Threema
Android application version 1.3.

This paper begins with an introduction of the used cryptography
library in Section~\ref{sec:nacl} and continues with an explanation of
the used keys in Section~\ref{sec:keys}. Before data can be
exchanged with the Threema server, a connection has to be
established. This handshake will be described in
Section~\ref{sec:handshake}.
After a successful handshake server and
client can exchange data. The different kinds of messages are
discussed in Section~\ref{sec:data}.
Section~\ref{sec:intercept} will give
some advice on how to verify the described protocol by
looking at the traffic generated by the application.

\section{The NaCl library}\label{sec:nacl}
The protocol used by Threema is based on the NaCl library. On
\href{http://nacl.cr.yp.to/}{its project page} NaCl is being described
as:

\begin{quote}
  NaCl (pronounced ``salt'') is a new easy-to-use high-speed software library for network communication, encryption, decryption, signatures, etc. NaCl's goal is to provide all of the core operations needed to build higher-level cryptographic tools.
\end{quote}

NaCl provides the
$\mathit{crypto\_box}$ function, among other primitives, that can be used to exchange
authenticated messages between a sender and a recipient. Given a
message $m$, the ciphertext $c$ is produced by using the recipient's
public key $PK_r$, the senders secret key $SK_s$ and a
nonce $n$.

\begin{equation}
 c = \mathit{crypto\_box}(m, n, \mathit{PK}_r, \mathit{SK}_s)
\end{equation}

The recipient can decipher this message using the
$\mathit{crypto\_box\_open}$ function with his secret key $\mathit{SK}_r$,
the senders public key $\mathit{PK}_s$ and the nonce $n$ used to create the ciphertext.

\begin{equation}
m = \mathit{crypto\_box\_open}(c, n, \mathit{PK}_s, \mathit{SK}_r)
\end{equation}

In contrast to traditional public-key cryptosystems, like RSA, a
ciphertext generated using the $\mathit{crypto\_box}$ function can not
only be deciphered with the recipient's secret key. It can also be
deciphered using the secret key of the sender.  This property is the
result of $\mathit{crypto\_box}$ using \href{https://en.wikipedia.org/wiki/ECDH}{elliptic
  curve Diffie-Hellmann} internally to derive a shared secret between both keypairs.

\section{Short- and long-term keys}\label{sec:keys}

Each Threema client generates a keypair on its first launch. This
keypair is called the long-term keypair and consists of a public key
$\mathit{LPK}_c$ and a secret key $\mathit{LSK}_c$. The
long-term public key is linked to an 8 byte username by sending it to the
Threema server\footnote{This is done using the Threema REST API. The REST
  API is not a direct part of the communication protocol and will not
  be discussed in this paper.}.

The long-term public key has to be exchanged with other Threema
users. To send a message the client has to know the username and
associated long-term public key of the recipient. The details will be
discussed in Section~\ref{sec:messages}.

Whenever a Threema client wants to establish a connection with the
server it will generate another keypair, used only for a short amount of
time. This keypair is called the short-term keypair and consists of
the short-term public key $\mathit{SPK}_c$ and the short-term secret
key $\mathit{SSK}_s$. The details of this handshake are explained in
the next section.

\section{Handshake}\label{sec:handshake}

The protocol is based on TCP using the client-server model.  On top of a
TCP handshake, the client has to perform a cryptographic handshake with the
server before it can send messages. This handshake is similar to the
one in the \href{http://curvecp.org/index.html}{CurveCP}
protocol.  During the handshake, client and server will exchange, and
agree on, the following data:

\begin{itemize}
  \item Client short-term public key: $\mathit{SPK}_c \in \{0..255\}^{32}$
  \item Server short-term public key: $\mathit{SPK}_s \in \{0..255\}^{32}$
  \item Client nonce prefix: $\mathit{NP}_c \in \{0..255\}^{16}$
  \item Server nonce prefix: $\mathit{NP}_s \in \{0..255\}^{16}$
  \item Client username: $u \in \{A ..Z, 0 ..9\}^8$
  \item Client system data, such as the Threema version, platform and the Java runtime
\end{itemize}

The nonce prefix is used together with an 8 byte counter to generate a
unique 24 byte nonce for every message\footnote{This method is explained in
  \href{https://stackoverflow.com/questions/13663604/questions-about-the-nacl-crypto-library/13663945\#13663945}{this}
  Stack Overflow discussion.}. I will be using the notation
$n_{xi} = \mathit{nonce}(\mathit{NP}_x, i)$ to express that the nonce $n_{xi}$ is
based on the nonce prefix $\mathit{NP}_x$ and the counter
value $i$.

\subsection{Client Hello}

The connection to the server is initialized by the client sending the
Client Hello packet (Figure~\ref{fig:hello-packet}).  Before sending this
packet, the client generates a short-term keypair using the
\href{http://nacl.cr.yp.to/box.html}{crypto\_box\_keypair} function
from the NaCl library.
It will also generate a 16 byte random nonce prefix
$\mathit{NP}_c$, that is used to generate unique nonces later on.

\begin{figure}[h]
  \centering
  \begin{bytefield}{16}
    \bitheader{0,15} \\
    \begin{leftwordgroup}{32 byte}
      \wordbox[lrt]{2}{client short-term public key $\mathit{SPK}_c$}
    \end{leftwordgroup} \\
    \wordbox{1}{client nonce prefix $\mathit{NP}_c$}
  \end{bytefield}
  \caption{The Client Hello packet}
  \label{fig:hello-packet}
\end{figure}

\subsection{Server Hello}

After receiving the Client Hello, the server will respond with a
Server Hello packet (Figure~\ref{fig:server-hello-packet}). Before
sending this packet, it also generates a short-term keypair and a 16 byte
random nonce prefix $\mathit{NP}_s$.

\begin{figure}[h]
  \centering
  \begin{bytefield}{16}
    \bitheader{0,15} \\
    \wordbox{1}{server nonce prefix $\mathit{NP}_s$} \\
    \begin{leftwordgroup}{64 byte}
      \wordbox{4}{$\mathit{ciphertext}_a$}
    \end{leftwordgroup}\\
  \end{bytefield}
  \caption{The Server Hello packet}
  \label{fig:server-hello-packet}
\end{figure}

Along with the nonce prefix $\mathit{NP}_s$ the server will send the
$\mathit{ciphertext}_a$. The contents of that ciphertext are the
server's short-term public key $\mathit{SPK}_s$ and the client's nonce
prefix $\mathit{NP}_c$. The client's nonce prefix is included to confirm
that the server received to correct one. Because the client
does not yet know about the server's short-term key, the
server's long-term secret key has to be used to create the ciphertext
(Equation~\ref{eq:server_hello}).

\begin{equation}\label{eq:server_hello}
  \mathit{ciphertext}_a = \mathit{crypto\_box}(\mathit{SPK}_s + \mathit{NP}_c, \mathit{nonce}(\mathit{NP}_s, 1), \mathit{SPK}_c, \mathit{LSK}_s)
\end{equation}

\subsection{Authentication}

When the server has introduced itself with the Server Hello packet,
the client sends the encrypted authentication packet
(Figure~\ref{fig:authentication-packet} and
Figure~\ref{fig:ciphertext_c-authentication}). Its purpose is to
authenticate the user, confirm the server's nonce prefix and send some
data about the client's system, like the used Threema version. The
authentication packet is the first packet that is being encrypted
using both short-term keys.
(Equation~\ref{eq:authentication_data}).

\begin{equation}\label{eq:authentication_data}
  \mathit{ciphertext}_b = \mathit{crypto\_box}(\mathit{authentication\_data}, \mathit{nonce}(\mathit{NP}_c, 1), \mathit{SPK}_s, \mathit{SSK}_c)
\end{equation}

\begin{figure}
  \centering
  \begin{bytefield}{16}
    \bitheader{0,15} \\
    \begin{leftwordgroup}{144 byte}
      \wordbox[lrt]{1}{$\mathit{ciphertext}_b$} \\
      \skippedwords \\
      \wordbox[lrb]{1}{}
    \end{leftwordgroup}
  \end{bytefield}
  \caption{The Authentication packet. It is fully encrypted with the short-term keys.}
  \label{fig:authentication-packet}
\end{figure}

\begin{figure}
  \centering
  \begin{bytefield}{32}
    \bitheader{0,7,8,15,16,23,24,32} \\
    \begin{leftwordgroup}{128 byte}
      \bitbox{8}{username $u$} & \bitbox[ltr]{24}{system data} \\
      \bitbox[lr]{8}{system data (cont.)} & \bitbox{16}{server nonce prefix $\mathit{NP}_s$} & \bitbox{8}{$\mathit{random\_nonce}$} \\
      \bitbox{16}{$\mathit{random\_nonce}$ (cont.)} & \bitbox{16}{$\mathit{ciphertext}_c$} \\
      \bitbox{32}{$\mathit{ciphertext}_c$ (cont.)}
    \end{leftwordgroup}
  \end{bytefield}
  \caption{Structure of the $\mathit{authentication\_data}$ that is being encrypted as $\mathit{ciphertext}_b$}
  \label{fig:ciphertext_c-authentication}
\end{figure}

With the $\mathit{ciphertext}_c$ inside
$\mathit{authentication\_data}$ (Equation~\ref{eq:username_authentication}) the client is verifying that it
possesses the private key $\mathit{LSK}_c$ that belongs to the long-term public key $\mathit{LPK}_c$ and is linked to the
username $u$.

\begin{equation}\label{eq:username_authentication}
  \mathit{ciphertext}_c = \mathit{crypto\_box}(\mathit{SPK}_c, \mathit{random\_nonce}, \mathit{LPK}_s, \mathit{LSK}_c)
\end{equation}

\subsection{Acknowledgment}

In the last step of the handshake, the server acknowledges that the
authentication data provided by the client is valid. The 32 byte
packet is fully encrypted and can be decrypted as shown in
equation~\ref{eq:acknowledgment}. Its content is meaningless and
contains only zeros. When the packet can be decrypted, the handshake
is completed and a connection with the Threema server has been established.

\begin{equation}\label{eq:acknowledgment}
  \mathit{message}_d = \mathit{crypto\_box\_open}(\mathit{ciphertext}_d, \mathit{nonce}(\mathit{NP}_s, 2), \mathit{SPK}_s, \mathit{SSK}_c)
\end{equation}

\section{Exchanging data with the server}\label{sec:data}

After the handshake is completed, the client and the server are allowed to exchange data.
The data is structured into different kinds of \textit{data packets} that will be described in the next section.
Whenever a number is used in any of the data packets, it will be encoded  using the
\href{https://en.wikipedia.org/wiki/Endianness}{little-endian byte
  order}.
For example the
\href{https://en.wikipedia.org/wiki/Integer\_%28computer\_science%29#Short\_integer}
{unsigned short integer} $2342_{10}$ is encoded as
$0926_{16}$ using big-endian and as $2609_{16}$ using
little-endian. Since big-endian is the standard format on most client
platforms, numbers have to be converted between little- and big-endian in most of the cases.

\subsection{General format of a data packet}

A data packet is the general format for data exchanged between the server and the client.
Each data packet is prefixed with a two byte length field.
It contains an unsigned short integer encoded as little-endian.
After reading those two bytes, the receiver knows how
many bytes it has to read from the sender in total.

Each data packet is encrypted and can be decrypted by using the appropriate
short-term keys for the client and server.
Depending on who sent packet, the next nonce generated by the server's
or the client's nonce prefix has to be used.

\begin{table}[h]
  \centering
  \begin{tabular}{|c |c|}
    \hline
    Type          & Description              \\ \hline
    \texttt{0x01} & Sending message          \\ \hline
    \texttt{0x02} & Delivering message       \\ \hline \hline
    \texttt{0x81} & Server Acknowledgment    \\ \hline
    \texttt{0x82} & Client Acknowledgment    \\ \hline \hline
    \texttt{0xd0} & Connection established   \\ \hline
  \end{tabular}
  \caption{Data packet types}
  \label{tb:data_packet_types}
\end{table}

The first four bytes of the decrypted data packet are used to encode
its type.  Table~\ref{tb:data_packet_types} shows some of the defined
types.
Most data packets will contain additional information besides its type identifier.
One exception is the data packet \texttt{0xd0}. It will be sent by the server after the
connection is successfully established and all queued messages have been sent to the client.
It is used by the client to know when the user can start sending messages. The meaning of the
remaining data packet types will be explained in the following section.

\subsection{Sending and receiving messages}\label{sec:messages}

The core purpose of the Threema protocol is to exchange messages with
other users. Data packets with the type \texttt{0x01} and \texttt{0x02} are used for this
(Figure~\ref{fig:message-packet}). The difference between the two
types is that \texttt{0x01} is used for outgoing and \texttt{0x02} is used for incoming
messages. Messages are used for different purposes in the Threema
protocol. In order to differentiate between the various kinds of
messages, each message starts with a one byte message type
identifier.  Table~\ref{tb:message_types} lists some of them.

\begin{table}[h]
  \centering
  \begin{tabular}{|c |c|}
    \hline
    Type          & Description              \\ \hline
    \texttt{0x01} & Text message             \\ \hline
    \texttt{0x80} & Message delivery receipt \\ \hline
    \texttt{0x90} & User typing notification \\ \hline
  \end{tabular}
  \caption{Message types used inside the data packet \texttt{0x01} and \texttt{0x02}}
  \label{tb:message_types}
\end{table}

Each Threema message has a unique identifier in order to be referenced
by the server and by the recipient.  Figure \ref{seqmessage} shows an
example of a message exchange between two users and the server: After
a user composed a message it will be sent to the server. The server will
acknowledge that it has received the message.  When the message has
been delivered to the recipient, its delivery will be acknowledged,
by generating a new message containing the delivery receipt (\texttt{0x80}).

\begin{figure}
  \centering
  \begin{sequencediagram}

    \tikzstyle{inststyle}=[rectangle, anchor=west, minimum width=4.5cm, minimum height=0.8cm, fill=white]
    \newthread{client1}{$Client_a$}
    \newthread{server}{$Server$}
    \newthread{client2}{$Client_b$}

    \mess{client1}{Send $m_1$: ``Hello''}{server}
    \mess{server}{Acknowledgment $m_1$}{client1}

    \mess{server}{Deliver $m_1$}{client2}
    \mess{client2}{Acknowledgment $m_1$}{server}

    \mess{client2}{Send $m_2$: receipt for $m_1$}{server}
    \mess{server}{Acknowledgment $m_2$}{client2}

    \mess{server}{Deliver $m_2$}{client1}
    \mess{client1}{Acknowledgment $m_2$}{server}
  \end{sequencediagram}
  \caption{Sending a message ($m_1$) and confirming its delivery by the recipient ($m_2$)}
  \label{seqmessage}
\end{figure}

Acknowledgments are data packets sent after the client or server have
received a message packet. Each acknowledgment packet includes the related message id and
the sender of the message. The identifier \texttt{0x81} is used by the server to
acknowledge that it has queued a message. The client uses the identifier
\texttt{0x82} to acknowledge that is has received one. If the server does
not receive an acknowledgment, it will attempt to re-transmit the
message. The same applies to the client if the server does not acknowledge the queuing of a message.
Figure~\ref{fig:message-ack} shows the structure of an acknowledgment packet.

\begin{figure}[h]
  \centering
  \begin{bytefield}{16}
    \bitheader{0, 3, 4, 11, 12, 19} \\
    \bitbox{4}{0x81\\0x82} & \bitbox{8}{sender identity} & \bitbox{8}{message id}
  \end{bytefield}
  \caption{Acknowledgment packet}
  \label{fig:message-ack}
\end{figure}

The structure of the message packet is show in Figure~\ref{fig:message-packet}.
The client will used \texttt{0x01} as a message identifier to send a
message and the server will use \texttt{0x02} to deliver a message.  Sender
and recipient are filled with the 8 byte identity of the related user.
The message id is randomly generated.  The $\mathit{time}$ field
encodes the current UTC time as an unsigned integer in little-endian
byte order.  The meaning of the four bytes in the $dunno$
field is not yet known. It contains zeros in most cases.
A fully random nonce will be used to encrypt the actual message content.

The server might know about the contents of a message by looking at its length.
To prevent this each message will be extended using
\href{https://en.wikipedia.org/wiki/Padding_%28cryptography%29#PKCS7}{PKCS7
  padding} before its encrypted. For example the ``user starts typing'' message has
always the same length of four bytes ($9000$).  Using PKCS7
padding with a random number of padding bytes (i.e. 5 bytes), the
message becomes longer ($90000505050505$) and the content is no longer guessable by looking at its ciphertext.

\begin{figure}
  \centering
  \begin{bytefield}{32}
    \bitheader{0, 3, 4, 11, 12, 19, 20, 27,28, 31} \\
    \bitbox{4}{0x01\\0x02} & \bitbox{8}{sender} & \bitbox{8}{recipient} & \bitbox{8}{message id} & \bitbox{4}{$time$} \\
    \bitbox{4}{dunno} & \bitbox{8}{pubkey owner} & \bitbox{20}{padding} \\
    \bitbox{4}{pad.} & \bitbox{24}{nonce} & \bitbox[lrt]{4}{} \\
    \bitbox[lrb]{32}{$\mathit{ciphertext}_n$ (variable length)}
  \end{bytefield}
  \caption{A message packet}
  \label{fig:message-packet}
\end{figure}

\section{Decrypt client-server communication}\label{sec:intercept}

If you want to verify the protocol described in this paper, you need
to know the client's short-term secret key $\mathit{SSK}_c$, the
server's short-term public key $\mathit{SPK}_s$ as well as the nonce
prefixes $\mathit{NP}_c$ and $\mathit{NP}_s$. In order to calculate
the counter values used to generate nonces by the client and the
server you either need to have all exchanged packets or guess the correct value by using brute force.

If you have all exchanged packets, another way of deciphering the
transport encryption is to use the server's long-term public key
$\mathit{LPK}_s$ together with the client's short-term secret key
$\mathit{SSK}_c$. Using both keys it is possible to decipher the whole
handshake and extract all session parameters.

The method used to gain these keys during this analysis was to modify
the app and include additional logging output.

Once you removed the transport encryption used between client and
server you can read the payload of every message (see section \ref{sec:messages} for an example).
The Threema
\href{https://threema.ch/validation/}{Validation Logging} feature and
the related programs can be used to compare the logged data and
the intercepted message contents then.

\section{Summary}

I did not describe the protocol in every possible detail and instead
focused on providing enough information to understand the basics. I
hope this analysis is useful to you and would love to hear your
thoughts.

I did this analysis out of my private interest. My intention in writing
this paper was to improve the trust in Threema.  I invite you to judge
about its safety on your own.

If you found a mistake or have any comments about this paper I'd be
happy if you contact me via email\footnote{My E-mail address and PGP key are on the first page.
  Please don't contact me
  regarding the extraction of Threema keys and servers. As already mentioned they're
  not part of this paper and are sold by Kasper Systems GmbH.}.
Please contact \href{https://threema.ch/en/contact.html}{Kasper
  Systems GmbH} directly if you have found any (potential) weaknesses in
their protocol.

\end{document}
